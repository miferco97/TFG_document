\chapter{Conclusiones y trabajo futuro}

Durante el transcurso de este proyecto se ha conseguido desarrollar una plataforma, que permite estudiar los distintos algoritmos de control, con los que se puede estabilizar a un cuadricóptero. A continuación se hablara sobre las conclusiones que se han extraído durante este proceso y las posibles líneas de mejora que se podrían realizar en un futuro. En los apéndices se encuentra la planificación , el presupuesto y el impacto medioambiental del trabajo. 

\section{Conclusiones}

Los objetivos principales que se plantearon al comienzo del proyecto consistían en: desarrollar un autopiloto propio capaz de ejecutar distintos algoritmos de control para cuadricópteros, desarrollar una plataforma de vuelo para poder probar este autopiloto y los algoritmos de forma segura e investigar sobre la posibilidad de emplear distintos algoritmos de aprendizaje por refuerzo para el control de actitud de la aeronave. A continuación desgranaremos las conclusiones que se han extraído de cada uno de estos objetivos y las posibles mejoras que se pueden realizar de cada uno.

\subsubsection{Autopiloto}
Se ha conseguido desarrollar un autopiloto funcional, capaz de estimar el estado de la aeronave y estabilizarse, cerrando un bucle interno de control y generando los comandos necesarios para controlar los motores. En este autopiloto se ha integrado la etapa de potencia necesaria para que sea posible alimentar al mismo directamente desde la batería, por lo que integra en una única PCB una gran cantidad de funcionalidades que no son habituales en los autopilotos comerciales, como por ejemplo la conexión WiFi, lo que ha permitido recabar los datos en tiempo real de forma sencilla. Sin embargo, las pistas que alimentan los motores a través de placa no se encuentran lo suficientemente ventiladas, por lo que se calientan demasiado.

\subsubsection{Plataforma de vuelo}
La plataforma de vuelo, ha permitido llevar a cabo los experimentos realizados de una forma segura y controlada. Debido a que la inmensa mayoría de las piezas han sido fabricadas mediante técnicas de impresión 3D, ha sido posible realizar variaciones de la plataforma de forma rápida y sencilla. Es por ésto que ,  se han diseñado varios componentes intercambiables, para poder modificar la configuración de los experimentos, lo que ha sido crucial para poder realizar el ajuste los parámetros del PID en las pruebas reales. El inconveniente que tiene la impresión 3D empleando PLA, es la fragilidad de algunas piezas, como por ejemplo, la unión esférica del banco de pruebas.

\subsubsection{Algoritmos basados en aprendizaje por refuerzo}

Uno de los retos que más tiempo han requerido, ha sido conseguir que los algoritmos propuestos, convergieran a una política óptima. Ha sido un proceso iterativo que ha tomado mucho tiempo hasta que se consiguió la convergencia del primer algoritmo. Posteriormente se realizaron muchas modificaciones de los hiperparámetros con el ánimo de mejorar el comportamiento del agente. Finalmente se ha conseguido aplicar varios algoritmos del estado del arte, consiguiendo , en algunos casos como en el del TRPO, muy buenos rendimientos.

\section{Trabajo futuro}

Esta plataforma, puede tener un gran interés para labores investigadoras y docentes. Para que pueda ser empleable de forma cómoda en estos ámbitos, sería posible simplificar el autopiloto, reduciendo así su tamaño y coste, ademas de reducir las dimensiones de la aeronave y el banco de pruebas, empleando por ejemplo motores DC, los cuales son mucho más pequeños y permiten controlarse de forma más sencilla. Si se reduce el tamaño, sería posible emplear varias de estas plataformas en los laboratorios de institutos y universidades para enseñar teoría de control. Además de reducir el tamaño, sería conveniente diseñar un interfaz más sencillo, pudiéndose integrar con programas como Matlab y Simulink, ésto mejoraría la usabilidad del mismo.

De cara a la investigación, sería conveniente realizar un modelo preciso de la aeronave en concreto, en vez de usar el modelo de otro cuadricóptero. Cuanto más parecido sea el modelo que se emplea en simulación al modelo real de la aeronave, menos se diferenciarán los comportamientos observados en simulación y el comportamiento real. En el desarrollo de algoritmos basados en aprendizaje por refuerzo, un aspecto muy relevante es el salto entre el entorno simulado y el entorno real, si el modelo real es lo suficientemente distinto al simulado, los comportamientos pueden variar enormemente. Es por esto que sería conveniente realizar una modelización dinámica completa de la aeronave.

En cuanto a los algoritmos, existen muchas maneras distintas de poder emplear el aprendizaje automático al campo del control de cuadricópteros, sería interesante replantearse la forma en la que los algoritmos actúan, para mejorar el rendimiento de estos algoritmos de forma que sobrepasen a los algoritmos de control clásico.Por ejemplo, en vez de generar un controlador en posición basado en RL, sería interesante diseñar un controlador en velocidad y sobre éste emplear un regulador P, basándose en la filosofía del controlador en cascada.  







	